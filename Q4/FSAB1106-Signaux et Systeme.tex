\documentclass[a4paper,12pt,oneside]{report}
\usepackage{titlesec}
\usepackage[svgnames]{xcolor}
%\usepackage[usenames,dvipsnames]{color}
\usepackage[top=0cm,bottom=0cm,left=0cm,right=0cm]{geometry}
%\usepackage{fullpage}
\usepackage[utf8]{inputenc}
\usepackage[T1]{fontenc}
\usepackage{graphicx}
\usepackage{wallpaper}
\usepackage{wrapfig}
\usepackage{hyperref}
\usepackage[francais]{babel}
%\usepackage{color}
\usepackage{layout}
\usepackage{circuitikz}
\usepackage[squaren, Gray]{SIunits}
\usepackage{sistyle}
\usepackage[autolanguage]{numprint}

%Algo package
\usepackage[ruled,vlined]{algorithm2e}
\usepackage{algpseudocode}

\usepackage{array}
\usepackage{calc}

\hypersetup{pdfborder={0 0 0}}

\titleformat{\part}
{\centering\fontfamily{pag}\fontsize{30}{30}\selectfont}
{\fontfamily{pag}\fontsize{30}{30}\selectfont Partie \ \thepart \ - }
{0pt}
{}{}

%Modifie le format des chapitres
\titleformat{\chapter}
{\it \color{DodgerBlue} \fontfamily{pag}\fontsize{20.74}{20}\selectfont}
{\it \fontfamily{pag}\fontsize{20.74}{20}\selectfont Chapitre \ \thechapter \ - }
{0pt}
{}{}
\titlespacing{\chapter}{0pt}{0.5cm}{0.5cm}[0pt]

%Modifie le format des sections
\titleformat{\section}
{\color{LimeGreen}\fontfamily{pag}\fontsize{15}{15}\selectfont}
{\fontfamily{pag}\fontsize{15}{15}\selectfont \thesection }
{5pt}
{}{}
\titlespacing{\section}{1cm}{0.5cm}{0.25cm}[0cm]

%Modifie le format des sous-secions
\titleformat{\subsection}
{\color{DarkOrange}\fontfamily{pag}\fontsize{12}{12}\selectfont}
{\fontfamily{pag}\fontsize{12}{12}\selectfont \thesubsection}
{5pt}
{}{}
\titlespacing*{\subsection}{2cm}{0.1cm}{0.1cm}[0cm]

%Modifie l'espace avant les paragraphes
\setlength{\parindent}{0pt}
\makeatletter

\newcommand{\bigO}{$\mathcal{O}$}

% Commande créeant une page de titre
%	#1=Titre
%	#2=Sous-titre
%	#3=Auteur
%	#4=image
%	#5=couleur de fond
%	#6=couleur cadre titre,sous-titre,auteur
\newcommand{\titre}[6]
{
	\begin{document}
	\begin{titlepage}
		\pagecolor{#5} % met la couleur de la page
		\vspace*{0.6cm}\fontfamily{pag}\fontsize{20}{20}\selectfont{\begin{center}2015-2016\end{center}}\vspace*{0.3cm}	% insère l'année en haut de la page
		\vspace*{-0.1cm}
		\includegraphics[width=\paperwidth]{#4} % insère l'image
		\colorbox{#6}{	%boîte de couleur avec le titre, sous-titre et l'auteur
			\hspace*{1.2cm}
			\begin{minipage}{\textwidth-1.8cm}{
				\vspace*{1cm}
				\begin{flushleft}\bfseries
					\fontfamily{pag}\fontsize{35}{35}\selectfont{#1}	%Titre
					\vspace*{0.5cm}
				\end{flushleft}
				\begin{flushleft}
					\fontfamily{pag}\fontsize{30}{30}\selectfont{#2}	%Sous-Titre
				\end{flushleft}
				\vspace*{1cm}
				\begin{flushright}
					\fontfamily{pag}\fontsize{20}{20}\selectfont{#3}	%Auteur
				\end{flushright}}
			\vspace*{1cm}
			\end{minipage}}
			\begin{center}
				\vfill % repli le reste de la page
				\fontfamily{pag}\fontsize{20}{20}\selectfont{\today} 	%Date
			\end{center}
	\end{titlepage} % fin de la page de titre
	\color{black} % met la couleur du reste du texte à noir
	\pagecolor{white}	% met la couleur des pages suivante à blanc
	\newgeometry{top=1.5cm,bottom=1.5cm,left=1cm,right=1cm}	% modification des marges
	\tableofcontents	%tables des matières
}
\titre{FSAB - 1106\\Signaux et Systeme}{Synthèse}{Damien Deprez}{FSAB1106}{LightBlue}{LightGreen}
\chapter{Terminologie}
\section{Signal}
Un signal est une fonction d'une ou de plusieurs variables qui contient des informations à propos d'un phénomène physique.
\section{Système}
Un système est une entité qui manipule un ou plusieurs signaux présents à son entrée, et produit un ou plusieurs nouveaux signaux. Certains systèmes sont dit en boucle fermé comme sur la figure
\section{Analogique VS. Numérique}
\subsection{Analogique}
Un signal analogique est un signal ou le temps et les valeur de l'amplitude du signal ne sont pas discrétisées. Le traitement de ce signal est basé sur des Résistances, des Capacités, des Inductances, des Diodes, des Transistors. Il est toujours en temps réel.
\subsection{Numérique}
Un signal numérique  est un signal ou le temps et les valeurs de l'amplitude du signal sont discrétisée. Ce type de signal permet d'avoir des fonctions d'addition, de multiplication et de mémoire. Il n'est pas toujours en temps réel.

\end{document}

